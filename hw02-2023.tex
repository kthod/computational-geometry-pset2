\documentclass[12pt]{article}
\usepackage{a4wide}
\usepackage{color, amssymb}
\usepackage[margin=1in]{geometry}
\usepackage[document]{ragged2e}
\usepackage[table]{xcolor}
\usepackage{multirow}
\setlength{\arrayrulewidth}{0.5mm}
\setlength{\tabcolsep}{16pt}
\renewcommand{\arraystretch}{1.9}
\usepackage[english]{babel}
\usepackage{mathtools}
\usepackage{ragged2e}
\renewcommand{\baselinestretch}{1.5}
\input{epsf}
\usepackage{float}
\usepackage{graphicx}
\usepackage{caption}
\usepackage{subcaption}
\usepackage{algorithm}
\usepackage[noend]{algpseudocode}
\usepackage[math]{cellspace}
\cellspacetoplimit = 3pt
\cellspacebottomlimit = 3pt

\begin{document}



1\\

    Prove the Meister-Gauss formula for the signed area of a simple $k$-gon $(p_0,p_1,\ldots,p_{k-1},p_k)$ (where $p_0=p_k$), that is,
    \[  E = \frac{1}{2} \sum_{i=0}^{k-1} p_i \times p_{i+1} \]
    
    Hint: Prove the formula for $k=3$ directly from definition, and use that for the case $k>3$.

2\\

    Prove that there are only five types of convex, regular polyhedra in $R^3$ (the so-called Platonic solids).


3\\

    You are given a polygonal area with a boundary of $n$ vertices and with $h$ holes. The area is not necessarily connected. Compute the number of diagonals and the number of triangles for any triangulation.

4\\


    Prove that a  polygonal gallery with $n$ vertices and $h$ holes can be guarded by at most $\lfloor \frac{n+2h}{3} \rfloor$  guards.



4\\
Present an efficient algorithm which, given a set $P = \{p_1 , \ldots , p_n \}$ of $n$
points on the integer grid $\mathbb{Z}^2$, computes the polygon of minimum perimeter that 
encloses these points, subject to the condition that the sides of this enclosure can be horizontal, 
vertical, or sloped at ±45◦ (see Figure). 

\textbf{Hint:} $O(n)$ time is possible, but $O(n \log n)$ time acceptable.
4\\
Prove that the enclosure produced by your algorithm has the minimum perimeter. Note that
there may generally be many enclosures with the same perimeter, and your algorithm may
output any of them.)


% \begin{center}
% \begin{tikzpicture}[
%     scale=0.5,
%     dot/.style={circle,fill,inner sep=1.5pt}
%     ]

%     \draw[step=1.0,black,thin,dotted]  (-1.5,-1.5) grid (6.5,6.5);
%     \foreach \nx/\ny in {%
%         1/0, 
%         0/2, 
%         1/1,1/3,1/5,
%         2/2,2/3,2/5,
%         3/2,
%         4/1,4/3,
%         5/3
%     }
%         \node[dot] at (\nx,\ny) {};

%     \draw[thick,black,fill=cyan,fill opacity=0.1] (0,2) -- (1,1) -- (1,0) -- (2,1) -- (4,1) -- (5,2) -- (5,3) -- (4,3) -- (2,5) -- (1,5) -- (1,3) -- cycle;
    
% \end{tikzpicture}
% \end{center}



\end{document}
$\lambda_1 < \lambda_2$